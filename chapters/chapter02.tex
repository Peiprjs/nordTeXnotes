% =-=-=-=-=-=-=-=-=-=-=-=-=-=-=-=-=-=-=-=-=-=-=-=-=-=-=-=-=-=-=-=-=-= Chapter 02
\chapter{Typesetting Mathematics}

This chapter will focus on inserting mathematics into your paper. \LaTeX\ comes 
with basic mathematics support out of the box, so we will need to 
incorporate some additional packages to enhance the range of formulas that can 
be included in our document.  You can think of it as adding the functionality 
of an equation editor into your current word processor. As you become more 
familiar with the syntax, you will quickly appreciate being able to type
your formulas with the same ease you write text.
\LaTeX\ comes with two modes that are of interest to us:
\begin{enumerate}
  \item normal and
  \item math, which has the two styles
  \begin{enumerate}
    \item inline and
    \item display.
  \end{enumerate}
\end{enumerate}
Normal mode is the default mode and where you will write the non-mathematical text
of your document.  If you want to include expressions and equations,
then you are going to need to use either an inline or display math mode, 
which are environments specific for mathematics.  However, we are going to
want to include more advanced mathematical formatting such as multiline 
equations and for this we are going to include some additional packages
including those from the \AmS, American Mathematics Society, and the mathtools\cite{CTANmathtools}
packages which will bring some enhancements to the \AmS\ math package\cite{CTANamsmath}.  Don't worry,
these packages are loaded by default when using this style file; however, 
you should be aware which packages are being used so that you know which 
documentation to reference for help beyond this quick start up guide.

To get us started, I am going to briefly introduce both the inline and 
display math environments that are available.  Then I will provide some of the 
more common features available from the \AmS math package.  Once we have 
established the environments in which we can write mathematics, we can look
at how we can format special characters  and symbols such as operators, 
operands and function names.  To make our lives a bit easier, we will include
some additional packages along the way and I will share some of the custom
macros that are available through the mhomath.sty file.

% =-=-=-=-=-=-=-=-=-=-=-=-=-=-=-=-=-=-=-=-=-=-=-=-=-=-=-=-=-=-=-=-=-=-= SECTION
\section{Inline Mathematics}
 In order for the code
that you write to be interpreted by the compiler as mathematics, you need
to toggle between \defw{normal mode} and \defw{inline math mode} by using the
math environment.
\begin{docEnvironment*}[doclang/environment content=mathematics content goes here]{math}{}{}
\begin{dispExample}
Solve all equations of the form 
\begin{math} 
  ax^2 + bx + c = 0 
\end{math}
for 
\begin{math} 
  x \in \mathbb{R} 
\end{math} 
by completing the square.
\end{dispExample}
From the above example, we can see that the code is verbose and difficult to
read.  Thankfully, this environment has a shortcut notation that should 
\textbf{ALWAYS} be used.  The macro \cs{(} is used to toggle into inline math mode 
and the macro \cs{)} is used to toggle back to normal mode. 
\begin{dispListing}
< normal mode text > \( < math environment code > \) < normal mode text > 
\end{dispListing}
We can now revisit our original example using this simplified notation.
\begin{dispExample}
Solve all equations of the form \( ax^2 + bx + c = 0 \) 
for \( x \in \mathbb{R} \) by completing the square.
\end{dispExample}
It is helpful to know that spaces in math mode are ignored by the 
compiler, which means we can be generous with how we use spaces to format
our code such that it is easier to parse. As your expressions become more 
complex, the additional spaces in your code should improve its legibility.  
Well, at least that has been my experience.  Let's consider our
example without the extra spaces.
\begin{dispExample}
Solve all equations of the form \(ax^2+bx+c=0\) for
\(x\in\mathbb{R}\) by completing the square.
\end{dispExample}
The shortcut inline math environment, \cs{(}\meta{math environment code}\cs{)}, that we 
have been using is from \LaTeX.  However, there is a plain \TeX\ , shortcut
for the math environment that remains popular today which is using the dollar
sign character as the delimiter.
\begin{dispListing}
< normal mode text > $ < math environment code > $ < normal mode text > 
\end{dispListing}
I prefer the \cs{(} \meta{math environment code} \cs{)} notation as it can be helpful
in resolving errors.  However, if you are new to using \LaTeX\,, then you might
want to use the dollar sign delimiters as they have the advantage of making 
your code slightly easier to read and faster to type.  At this point in the 
development of \LaTeX\, as a language, it is really a matter of personal 
preference.  Just be consistent!
\begin{dispExample}
Solve all equations of the form $ ax^2 + bx + c = 0 $ for 
$ x \in \mathbb{R} $ by completing the square.
\end{dispExample}
A small thought on changing between the two different shortcut notations. It 
would be possible to do a quick search and replace from the \LaTeX\, notation to 
the \TeX\, notation, but not vice-versa. 
\end{docEnvironment*}

% =-=-=-=-=-=-=-=-=-=-=-=-=-=-=-=-=-=-=-=-=-=-=-=-=-=-=-=-=-=-=-=-=-=-= SECTION
\section{Display Style Math Mode}

There will be times when you will want to emphasize or isolate an expression
by centering it on the page and for this we are going to be using the 
\defw{display style math mode} environment. This environment is restricted to 
a single line of output like an equation.

\begin{docEnvironment*}[doclang/environment content=mathematics content goes here]{displaymath}{}{}
\begin{dispExample}
All the real number values of \( x \) that satisfy the quadratic equations
of the form \( ax^2 + bx + c = 0 \) where \( a, b, c \in \mathbb{R} \) 
can be found using the quadratic formula,
\begin{displaymath}
  x = \frac{-b \pm \sqrt{b^2 - 4ac}}{2a}.
\end{displaymath}
\end{dispExample}
Just like the inline math environment, you would \textbf{ALWAYS} use the 
\LaTeX\,pair of \cs{[} \meta{math environment code} \cs{]} delimiters as the 
shortcut. Rather than write the delimiters and mathematical expression on the same 
line as we did in inline math mode, we give each delimiter it's own line 
and place the math on the line in between the delimiters. We could express the
quadratic formula from our previous examples as:
\begin{dispExample}
\[
  x = \frac{-b \pm \sqrt{b^2 - 4ac}}{2a}.
\]
\end{dispExample}
It is very likely that you will come across the the double dollar sign 
shortcut notation for the display math environment, but it should 
\textbf{NEVER} be used as it will most likely bring up issues at some point. 
\begin{dispExample}
$$
  x = \frac{-b \pm \sqrt{b^2 - 4ac}}{2a}.
$$
\end{dispExample} 
If you are looking to include multiple lines in display math mode, then
you might be tempted to use consecutive display style math mode 
environments, but you must resist as a better solution is available.
In summary, you should \textbf{NEVER} use consecutive display math mode 
environments, which means you should never write something like:
\begin{dispExample}
  \[ 
    2x^2 + 14x + 24 = 2( x^2 + 7x + 12 )
  \]
  \[ 
    = 2(x + 3)(x + 3) 
  \].
\end{dispExample}
\end{docEnvironment*}

% =-=-=-=-=-=-=-=-=-=-=-=-=-=-=-=-=-=-=-=-=-=-=-=-=-=-=-=-=-=-=-=-=-=-= SECTION
\section{Multiline Math Environments}

Like I said, there are specific environments that you can use to achieve multiline 
equations. This style file includes the package mathtools, which will give us 
multiple environments to achieve multiline equations and more.  The mathtools package
imports the \AmS\ math package and enhances its functionality.  This means if we are
using mathtools, then there is no need to import \AmS\ math.

For complete documentation I suggest you check the official documentation:
\begin{itemize}
  \item User’s Guide for the amsmath Package
  \item The mathtools package
\end{itemize}

Before we look at the using multiline equation environments, I want to point out
the single line math environment called equation, which is similar to the \LaTeX 
~display math environment 
 
\begin{docEnvironment*}[doclang/environment content=mathematics content goes here]{equation}{}{}
  The first environment that we are going to consider is very similar to the the \LaTeX ~display 
  math environment, but it also supports multiline equations.  By default the equations will 
  be numbered.
  \begin{dispExample}
    \begin{equation}
      x = \frac{-b \pm \sqrt{b^2 - 4ac}}{2a}.
    \end{equation}
    \end{dispExample}
\end{docEnvironment*}
In fact, we get something very close to the \LaTeX ~display math environment with 
the star version of this equation environment as the equation numbers are suppressed.
All these structured math environments have a star version that suppresses numbering,
so I will only include an example for the equation environment.
\begin{docEnvironment*}[doclang/environment content=mathematics content goes here]{equation*}{}{}
  \begin{dispExample}
    \begin{equation*}
      x = \frac{-b \pm \sqrt{b^2 - 4ac}}{2a}.
    \end{equation*}
    \end{dispExample}
  \end{docEnvironment*}
It's time to take a look at some of the multiline equation environments that are available
for us to use in our work. To change the form of the expression \( 2x^2 + 14x + 24 \) using 
the equation environment. 
\begin{dispExample}
  \begin{equation*}
    2x^2 + 14x + 24 = 2( x^2 + 7x + 12 ) = 2(x + 3)(x + 3)
    \end{equation*}
  \end{dispExample}

However, this might be easier to read if it was expressed using a multiline math environment.

It is possible include a split environment within the equation environment to achieve a
multiline equation with the option to include expression splitting as well.
\begin{docEnvironment*}[doclang/environment content=mathematics content goes here]{split}{}{}
  We will use the \verb!&! character to vertically \textit{align} each equation
  at the equal sign and two consecutive double backslash characters, \verb!\\! 
  break to the next line.
  \begin{dispExample}
    \begin{equation*}
      \begin{split}
        2x^2 + 14x + 24 
        &= 2( x^2 + 7x + 12 ) \\
        &= 2(x + 3)(x + 3) 
      \end{split}
    \end{equation*}
    \end{dispExample}
    It is even possible to split up long expressions!  The \cs{quad} macro is 
    used to include 4 spaces.
    \begin{dispExample}
      \begin{equation*}
        \begin{split}
          a
          &= b - c + d - e + f \\
          & \quad - g + h - i + k \\
          &= l + m \\
          &= n 
        \end{split}
      \end{equation*}
      \end{dispExample}
\end{docEnvironment*}
It might be the case that you have one very long expression that you would like 
to break up and this can be done with the multiline environment.  Notice that 
this does not require to be embedded in the equation environment.
\begin{docEnvironment*}[doclang/environment content=mathematics content goes here]{multiline}{}{}
  \begin{dispExample}
    \begin{multline*}
        a + b + c + d + e + f + g + h + i + j + k + l + m + n + o + p + q + r + s + t + u + v + w + x + y + z \\
        - (a + b + c + d + e + f + g + h + i + j + k + l + m + n + o + p + q + r + s + t + u + v + w + x + y + z ) 
    \end{multline*}
    \end{dispExample}
\end{docEnvironment*}
My preference is to have my equations to be vertically aligned at the equal sign.
However, you might want your equations to be centered on the page.  The gather 
environment will take care of this for you.
\begin{docEnvironment*}[doclang/environment content=mathematics content goes here]{gather}{}{}
  \begin{dispExample}
    \begin{gather*}
        a = b + c + d + e \\
        a = f + g \\
        a = h
    \end{gather*}
    \end{dispExample}
  Notice that there is not \verb!&! character being used for alignment in this environment.
\end{docEnvironment*}
It is now time to introduce the environment that I use for about 95 \% of the 
multiple line expressions needed in my work and it is called align.  
\begin{docEnvironment*}[doclang/environment content=mathematics content goes here]{align*}{}{}
\begin{dispExample}
  \begin{align*}
    2x^2 + 14x + 24 
    &= 2( x^2 + 7x + 12 ) \\
    &= 2(x + 3)(x + 3). 
  \end{align*}
\end{dispExample}
This environment allows you to comment you work using tags
\begin{dispExample}
  \begin{align*}
    2x^2 + 14x + 24 
    &= 2( x^2 + 7x + 12 ) \tag{factoring two}\\
    &= 2(x + 3)(x + 3). \tag{factoring the quadratic}
  \end{align*}
\end{dispExample}
There is no need for multiple align environments in a 
sentence since there is the \cs{intertext} macro.  If 
you want less space between the lines, then I can
suggest that you use the \cs{shortintertext} macro instead.
\begin{dispExample}
  Given \( x = \cos x \) and \( y = \sin x \), we can show
  \begin{align*}
    \shortintertext{using the Pythagorean identity}
    x^2 + y^2 &= 1 \\
    \intertext{substituting \(x \) and \(y \)}
    \left( \cos x \right)^2 + \left( \sin x \right)^2 &= 1 \\
    \shortintertext{alternative function squared form}
    \cos^2 x + \sin^2 &= 1.
  \end{align*}
\end{dispExample}
It is also possible to include two columns of equations
  \begin{dispExample}
  \begin{align*}
    a_{11}
    & =b_{11}
    &
    a_{12}
    & =b_{12}\\
    a_{21}
    & =b_{21}
    &
    a_{22}
    & =b_{22}+c_{22}
\end{align*}
\end{dispExample}
\end{docEnvironment*}
If you would like the columns to be fully justified, then 
there is an alternative environment called flalign.
\begin{docEnvironment*}[doclang/environment content=mathematics content goes here]{flalign*}{}{}
\begin{dispExample}
\begin{flalign*}
    a_{11}
    & =b_{11}
    &
    a_{12}
    & =b_{12}\\
    a_{21}
    & =b_{21}
    &
    a_{22}
    & =b_{22}+c_{22}
\end{flalign*}
\end{dispExample}
\end{docEnvironment*}

% =-=-=-=-=-=-=-=-=-=-=-=-=-=-=-=-=-=-=-=-=-=-=-=-=-=-=-=-=-=-=-=-=-=-= SECTION
\section{Cases Environment}

There are a couple cases where you might want to treat some part of your
formula as a block on the same line and the cases environment get the job done.
\begin{docEnvironment*}[doclang/environment content=mathematics content goes here]{cases}{}{}
\begin{dispExample}
\[
\abs{x}=\begin{cases}
  x & \text{ if \( x \ge 0 \) } \\
  -x & \text{ if \( x < 0 \) }
\end{cases}
\]
\end{dispExample}
This can also be used for systems of equations.
\begin{dispExample}
Solve the following system of equations for \( x \) and \( y \).
  \[
  \begin{cases}
    x + y = 12 \\
    2x - 3y = 9
  \end{cases}
  \]
  \end{dispExample}
\end{docEnvironment*}

%%% Local Variables:
%%% mode: latex
%%% TeX-master: "../main"
%%% End:
