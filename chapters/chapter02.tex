%----------------------------------------------------------------------------------------------------------------------------------------------------------%
\chapter{Physical experimentation}
\epigraph{A scientist in his laboratory is not a mere technician: he is also a child confronting natural phenomena that impress him as though they were fairy tales.}{\textit{Marie Curie}}
%----------------------------------------------------------------------------------------------------------------------------------------------------------%
\section{Description}
\paragraph{}This experiment is designed to detect and evaluate the prevalence of \emph{Staphylococcus aureus} in a sample of students from our school. The process used involves extracting a sample from underneath a subject's nails by swabbing, cultivating that sample, and then observing the results of said culture to determine the presence or not of \emph{Staphylococcus aureus} as part of the subject's resident bacterial flora. Each sampling iteration of the process took less than two minutes to complete. However, all the safety measures and actions taken need more time to be taken care of properly; as well as taking into account the fact that cultivating is not a task that can be done in just a day, often needing two to three to fully develop.
%----------------------------------------------------------------------------------------------------------------------------------------------------------%
\section{Protocol followed}
\paragraph{}The protocol followed was designed based on a similar protocol used in many university laboratories\cite{olearyPracticalHandbookMicrobiology1989}, modified to fit the needs of this research paper, peer-reviewed by Olga Sánchez, and uploaded to the Protocols.io platform, to make it easier to follow the days of that the experiment took place in. This protocol underwent 10 different revisions\cite{rocacugatStaphilococcusAureusSampling2022a}. It dictates the following steps: \newline\begin{wrapfigure}{r}{0.4\textwidth}\begin{center}\includegraphics[width=0.38\textwidth]{sampling.JPG}\end{center}\caption{Transferring a sample to the agar plate}\end{wrapfigure}
\begin{enumerate}[label=\arabic*)]
\item Prepare yourself for the experimentation: wash your hands, put on gloves, put on the lab coat, mask, and goggles. Wash your hands again (gloves still on). Set up the work area; the Bunsen burner should be turned on in such a way that it can cover an acceptable surface to work. Turn it on and try not to break sterility
\item Divide each Petri dish in 2 parts. A ruler should be used for this part. Get your subject to wash their hands and observe them. If the nails are extremely short, it may be worth it to take the sample nasally. If the hands don't seem clean enough, teach them proper hand washing techniques.
\item Note down their information, crack open a sterile swab pack, dip one of the swabs in Ringer solution and swab away at under their nails or nose. Then, populate the dish with this sample.
\item Incubate for 32-48h and observe the results.
\item Observe the bacteria under a microscope after a GRAM staining.
\end{enumerate}\newpage
%----------------------------------------------------------------------------------------------------------------------------------------------------------%
\section{Bill of materials}
\paragraph{}The materials used, as well as the quantities used, can be found in the following table. On the left, laboratory equipment and, on the right, reagents, staining agents, and consumables used:
\begin{center}\begin{figure}[H]\centering\includegraphics[width=0.90\textwidth]{BOM-1.png}\end{figure}\end{center}
%----------------------------------------------------------------------------------------------------------------------------------------------------------%
\section{Biosecurity and risk mitigation}
\paragraph{}Staph is considered a Biosecurity Level (BSL) 2 pathogenic bacteria\cite{cheungPathogenicityVirulenceStaphylococcus2021}. This means that it is associated with a human disease that can pose a moderate human health hazard. In a laboratory where BSL-2 pathogens are handled, usual lab rules should be followed (mechanical pipetting only, surgical hand-washing, prohibition of the consumption of food and drinks in the lab, proper PPE use… as well as avoiding splashes or aerosols, adhering biohazard warning signs present on all material used, as well as proper surface and material disinfection via the use of autoclave or proper alternative decontamination method.\newline
The risks associated with this bacterium were assessed following the protocol designated by the World Health Organisation, and proper security measures were followed at all times when handling biohazardous material. No incidents occurred during the research part of this project, and the protocol defined previous to the start was followed extremely closely. While the laboratory used may not be the most ideal type of laboratory for this type of research, it was certainly adequate to perform a research project like this one, especially after the temporary signage that was temporarily installed\cite{worldhealthorganizationLaboratoryBiosafetyManual2020}.\newpage
%----------------------------------------------------------------------------------------------------------------------------------------------------------%
\section{Results and analysis}
The results obtained can be found in the following raw data table:
\begin{center}\begin{figure}[H]\centering\includegraphics[width=0.80\textwidth]{RES-1.jpg}\end{figure}\end{center}
The data was then recounted and graphed into the following pie chart:
\begin{center}\begin{figure}[H]\centering\begin{subfigure}[b]{0.4\linewidth}\includegraphics[width=0.95\linewidth]{Data.png}\caption{Counts of the result cases.}\end{subfigure}\begin{subfigure}[b]{0.38\linewidth}\includegraphics[width=0.95\linewidth]{Pie.png}\caption{Pie graph of the result cases.}\end{subfigure}\caption{Data processed from results}\end{figure}\end{center}\vspace{-1.5em}
As we can see, almost 50\% of the samples taken tested positive for \emph{Staphylococcus aureus}, compared to the expected 30\%\cite{StaphylococcusAureusHealthcare2020}. We can, however, see in the UK's Public Health bactaeremia data that Staph infections have been on the rise lately, so it may not be a case of wrong data\cite{englandMSSABacteraemiaAnnual2021}. On top of that, both of my advisers, Olga and Margarita, have also found their experiments resulting in a higher prevalence than usual of this bacterium, and are finding cases that were once negative but turned positive in the last few years. Most of these results were not only confirmed by the highly-specific detection of the MSA plate, but also by taking the morphological observation into account, both of the colonies and microscopically.\begin{figure}[H]\centering\begin{subfigure}[b]{0.4\linewidth}\includegraphics[width=\linewidth]{microscope.JPG}\caption{\emph{Staphylococcus aureus} as seen below the microscope. x4000, GRAM staining}\end{subfigure}\begin{subfigure}[b]{0.4\linewidth}\includegraphics[width=\linewidth]{colonies.JPG}\caption{Colonies of \emph{Staphylococcus aureus} seen under a magnifying glass}\end{subfigure}\caption{Photographies of the results, as collected from my own experimentation (own data).}\end{figure}\paragraph{}There may be several reasons for the infection rate and thus natural prevalence to be increasing. One of them could be that since antibiotic abuse is growing with each passing year, the usual resident microbiota is getting killed, leaving more resources for Staph to thrive in that environment. To confirm this theory, we will look at the infection rates of a country that is facing extreme antibiotic abuse (the United States of America) and compare it to another that is controlling their antibiotics a bit better (the United Kingdom). The former have seen a 210\% increase in \emph{Staphylococcus aureus} cases since 2006. However, superfluous antibiotic prescriptions have increased by barely 1\%\cite{baggsEstimatingNationalTrends2016}. In the United Kingdom, they have seen a 160\% increase in \emph{Staphylococcus aureus} infections\cite{englandMSSABacteraemiaAnnual2021}, and their superfluous antibiotic prescriptions have gone down by 20\%. Even though this is very little data to extract conclusions from, there may be a correlation between these two factors.
\paragraph{}The other could be climate change. An increase of ambient temperatures could mean a more suitable breeding ground for this specific species and thus leading to a higher-than-usual prevalence. \emph{Staphylococcus aureus}' optimal breeding temperature is between 35\si{\celsius} and 37\si{\celsius}. The global average temperature has increased by 1,1\si{\celsius}\cite{gmsGMSAnnualGlobal2016} in the last 120 years. And \emph{Staphylococcus aureus} has a specific temperature growth curve, just like any other bacteria:\begin{figure}[H]\centering\begin{subfigure}[b]{0.4\linewidth}\includegraphics[width=0.95\linewidth]{tempcurve.png}\caption{\emph{Staphylococcus aureus} growth curve by temperature\cite{FigEffectTemperature2022}}\end{subfigure}\begin{subfigure}[b]{0.4\linewidth}\includegraphics[width=0.95\linewidth]{warm.png}\caption{Global warming plotted by the year\cite{GlobalWarming2010}}\end{subfigure}\caption{Graphs relating the temperature of growth of \emph{Staphylococcus aureus} and the increase of temperature of the Earth.}\end{figure}So, this correlation may not be completely incorrect, and in fact some scientists warn about an increased number of infectious diseases seeing a growth in numbers due to climate change. One only data point is not enough significant data, so further study is needed on this front.
