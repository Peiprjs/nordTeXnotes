%----------------------------------------------------------------------------------------------------------------------------------------------------------%
\chapter{Physical experimentation}
\epigraph{A scientist in his laboratory is not a mere technician: he is also a child confronting natural phenomena that impress him as though they were fairy tales.}{\textit{Marie Curie}}
%----------------------------------------------------------------------------------------------------------------------------------------------------------%
\section{Description}
\paragraph{}This experiment is designed to detect and evaluate the prevalence of \emph{Staphylococcus aureus} in a sample of students from our school. The process used involves extracting a sample from underneath a subject's nails by swabbing, cultivating that sample, and then observing the results of said culture to determine the presence or not of \emph{Staphylococcus aureus} as part of the subject's resident bacterial flora. Each sampling iteration of the process took less than two minutes to complete. However, all the safety measures and actions taken need more time to be taken care of properly; as well as taking into account the fact that cultivating is not a task that can be done in just a day, often needing two to three to fully develop.
%----------------------------------------------------------------------------------------------------------------------------------------------------------%
\section{Protocol followed}
\paragraph{}The protocol followed was designed based on a similar protocol used in many university laboratories\cite{olearyPracticalHandbookMicrobiology1989}, modified to fit the needs of this research paper. This protocol underwent 10 different revisions. It dictates the following steps: \newline\begin{wrapfigure}{r}{0.4\textwidth}\begin{center}\includegraphics[width=0.38\textwidth]{sampling.JPG}\end{center}\caption{Photograph of me populating a Petri dish. Source: own  }\end{wrapfigure}
\begin{enumerate}[label=\arabic*)]
\item Set up the work area; the Bunsen burner should be turned on in such a way that it can cover an acceptable surface to work. Turn it on and try not to break the sterile field.
\item Prepare yourself for the experimentation: wash your hands, proper PPE. Wash your hands again (with gloves on).
\item Divide each Petri dish in 2 parts. A ruler should be used for this part. Get your subject to wash their hands and observe them. If the nails are extremely short, it may be worth it to take the sample nasally. If the hands don't seem clean enough, teach them proper hand washing techniques.
\item Note down their information, crack open a sterile swab pack, dip one of the swabs in Ringer solution and swab under their nails or nose. Then, populate the dish with this sample following the zig-zag method for one of the halves of the dish.
\item Incubate for 32-48h and observe the results.
\item Observe the bacteria under a microscope after a GRAM staining.
\end{enumerate}
%----------------------------------------------------------------------------------------------------------------------------------------------------------%
\section{Results and analysis}
The results obtained can be found in the following raw data table:
\begin{center}\begin{figure}[H]\centering\includegraphics[width=0.80\textwidth]{RES-1.jpg}\end{figure}\end{center}
The data was then recounted and graphed into the following pie chart:
\begin{center}\begin{figure}[H]\centering\begin{subfigure}[b]{0.4\linewidth}\includegraphics[width=0.95\linewidth]{Data.png}\caption{Counts of the result cases.}\end{subfigure}\begin{subfigure}[b]{0.38\linewidth}\includegraphics[width=0.95\linewidth]{Pie.png}\caption{Pie graph of the result cases.}\end{subfigure}\caption{Data processed from results}\end{figure}\end{center}\vspace{-1.5em}
As we can see, almost 50\% of the samples taken tested positive for \emph{Staphylococcus aureus}, compared to the expected 30\%\cite{StaphylococcusAureusHealthcare2020}. We can, however, see in the UK's Public Health bactaeremia data that Staph infections have been on the rise lately, so it may not be a case of wrong data\cite{englandMSSABacteraemiaAnnual2021}. On top of that, both of the experts I emailed, to see if they had also seen an uptick in cases, also found their experiments resulting in a higher prevalence than usual of this bacterium, and found cases that were once negative but recently turned positive. Most of these results were not only confirmed by the highly-specific detection of the MSA plate, but also by taking the morphological observation into account, both of the colonies and microscopically.
\paragraph{}There may be several reasons for the infection rate and thus natural prevalence to be increasing. One of them could be that since antibiotic abuse is growing with each passing year, the usual resident microbiota is getting killed, leaving more resources for Staph to thrive in that environment. To confirm this theory, we will look at the infection rates of a country that is facing extreme antibiotic abuse (the United States of America) and compare it to another that is controlling their antibiotics a bit better (the United Kingdom). The former have seen a 210\% increase in \emph{Staphylococcus aureus} cases since 2006. However, superfluous antibiotic prescriptions have increased by barely 1\%\cite{baggsEstimatingNationalTrends2016}. In the United Kingdom, they have seen a 160\% increase in \emph{Staphylococcus aureus} infections\cite{englandMSSABacteraemiaAnnual2021}, and their superfluous antibiotic prescriptions have gone down by 20\%. Even though this is very little data to extract conclusions from, there may be a correlation between these two factors.
\begin{figure}[H]\centering\begin{subfigure}[b]{0.4\linewidth}\includegraphics[width=\linewidth]{microscope.JPG}\caption{\emph{Staphylococcus aureus} as seen below the microscope. x4000, GRAM staining}\end{subfigure}\begin{subfigure}[b]{0.4\linewidth}\includegraphics[width=\linewidth]{colonies.JPG}\caption{Colonies of \emph{Staphylococcus aureus} seen under a magnifying glass}\end{subfigure}\caption{Photographies of the results, as collected from my own experimentation (own data).}\end{figure}
\paragraph{}Let's compare the prevalence among different groups of subjects, starting with their gender.
\begin{center}\begin{figure}[H]\centering\begin{subfigure}[b]{0.4\linewidth}\includegraphics[width=0.95\linewidth]{gend_fem.png}\caption{Counts of the result cases in female-gendered individuals}\end{subfigure}\begin{subfigure}[b]{0.38\linewidth}\includegraphics[width=0.95\linewidth]{gend_male.png}\caption{Counts of the result cases in male-gendered individuals}\end{subfigure}\caption{Data processed from results}\end{figure}\end{center}\vspace{-1.5em}
As we can observe, there is a 10\% difference in prevalence between these two genders. In my opinion, this is due to a small sample size. As we can see, the first graph, which houses the samples from subjects who identify themselves as female, has a third type of result, which the other graph, which houses the samples from subjects who self-identify as male, does not. This is simply due to a problem with the plates, however, it may affect the final result. I believe there to not be any significant difference between the two analized genders.
\begin{center}\begin{figure}[H]\centering\begin{subfigure}[b]{0.4\linewidth}\includegraphics[width=0.95\linewidth]{age_young.png}\caption{Counts of the result cases in female-gendered individuals}\end{subfigure}\begin{subfigure}[b]{0.38\linewidth}\includegraphics[width=0.95\linewidth]{age_old.png}\caption{Counts of the result cases in male-gendered individuals}\end{subfigure}\caption{Data processed from results}\end{figure}\end{center}\vspace{-1.5em}
Only two age groups were studied, due to the fact that the samples fell mostly into one of these two categories. We can see that the samples from a subject older than 40 were twice as likely to test positive for \emph{Staphylococcus aureus} than those from subjects younger than 20. This may again be due to a small sample size. However, it may also be possible that the possibility of hosting this bacterium increases with age. In order to confirm this theory, more studies should be done, with a greater age gradient, as well as many more samples.
