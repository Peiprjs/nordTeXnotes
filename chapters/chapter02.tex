% =-=-=-=-=-=-=-=-=-=-=-=-=-=-=-=-=-=-=-=-=-=-=-=-=-=-=-=-=-=-=-=-=-= Chapter 02
\chapter{Typesetting Mathematics}

One of the main advantages of using \LaTeX\,is that it is relatively easy to
include mathematical statements into your documents.  In order for the code
that you write to be interpreted by the compiler as mathematics, you need
to escape from \defw{normal mode} into \defw{math mode} by using the
\verb!\(! delimiter, enter the code you wish to express as a mathematical
statement, then exit from \defw{math mode} back into \defw{normal mode} by
using the \verb!\)! delimiter. This \defw{math mode} is more specifically
known as inline mathematics.

% =-=-=-=-=-=-=-=-=-=-=-=-=-=-=-=-=-=-=-=-=-=-=-=-=-=-=-=-=-=-=-=-=-=-= SECTION
\section{Inline Mathematics}

Part of the steep learning curve associated with \LaTeX\,is getting to know the
math mode syntax.  We are going to start with an inline math example that
includes how to express a power, the command for is an element of the set and
how to format sets names using blackboard bold.

\begin{mhotexbox}
Solve all equations of the form \( ax^2 + bx + c = 0 \) for
\( x \in \mathbb{R} \) by completing the square.
\end{mhotexbox}

Spaces in \textbf{math mode} are ignored by the compiler, which means we can
be somewhat generous with space in our code to make it easier to parse.  I
generally like to leave space on both sides of:
\begin{enumerate}
\item explicit infix binary operators such as \( +, - , \times, \div \);
%\item binary relations such as \( = , \leq, \geq, \divides, >, < \);
  \item latex commands like \verb!\in, \mathbb{R}! and
  \item math mode toggle commands \verb!\(! and \verb!\)!.
\end{enumerate}
It is not required that you leave these spaces, but consider how cramped the
code looks without any spaces.
\begin{mhotexbox}
Solve all equations of the form \(ax^2+bx+c=0\) for
\(x\in\mathbb{R}\) by completing the square.
\end{mhotexbox}
As your expressions become more complex, the additional spaces in your code
should improve the legibility of your code.  Well, at least that has been my
experience. If you are new to \LaTeX\,, then you might want to use the
plain \TeX\,\verb!$! delimiters to toggle between \textbf{normal mode} and
\textbf{math mode}.  The advantage is that it is easier to read the code, but
can make finding errors more difficult to locate later on.  You can always
do a search replace to go from \verb!\( \)! to \verb!$$!, but the converse
is not possible.
\begin{mhotexbox}
Solve all equations of the form $ ax^2 + bx + c = 0 $ for $ x \in \mathbb{R} $
by completing the square.
\end{mhotexbox}
It is now time to introduce \defw{display style math mode}.

% =-=-=-=-=-=-=-=-=-=-=-=-=-=-=-=-=-=-=-=-=-=-=-=-=-=-=-=-=-=-=-=-=-=-= SECTION
\section{Display Style Math Mode}

There will be times when you will want to emphasise or isolate an expression
by centering it on the page and for this we are going to be using 
\defw{display style math mode}.  If your expression is only one line then
we can use the \verb!\[ foo \]! delimiters, which can formatted such that
each delimiter gets its own line and the expression is given on the line 
between them.  You will most likely come across the plain \TeX\,double
dollar sign, \verb!$$ foo $$!, delimiters and unlike the single dollar sign
as delimiters in inline mode, these are \textbf{NOT TO BE USED} to toggle
display style math mode.

\begin{mhotexbox}
All the real number values of \( x \) that satisfy the quadratic equations
of the form \( ax^2 + bx + c = 0 \) where \( a, b, c \in \mathbb{R} \) 
can be found using the quadratic formula,
\[
  x = \frac{-b \pm \sqrt{b^2 - 4ac}}{2a}.
\]
\end{mhotexbox}

If you are looking to include multiple lines in display math mode, then
you have a few environment options available.  You should not use 
consecutive display math mode delimiters.  The following is to be avoided.

\begin{mhotexbox}
  \[ 
    2x^2 + 14x + 24 = 2( x^2 + 7x + 12 )
  \]
  \[ 
    = 2(x + 3)(x + 3) 
  \].
\end{mhotexbox}

There are many different environments for doing mutliple line expressions
in \defw{display math mode}. I use the \texttt{align} environment for
about 95 \% of my mutliple line expressions needed and is dependent upon 
the amsmath package.  I really like how I can format my code such that 
each expression of the equation \textbf{can} be given its own line, 
which is very useful for finding errors and viewing long lines of code.
In the upcomming example you will see that I use the \verb!&! to align
each of the multiple lines with the \verb!=! sign and use \verb!\\!
for an equation line break.

\begin{mhotexbox}
Changing the form of the expression \( 2x^2 + 14x + 24 \) in inline math 
mode, could write \( 2x^2 + 14x + 24 = 2( x^2 + 7x + 12 ) = 2(x + 3)(x + 3) \).  
However, this might be easier for the reader if it was expressed in 
display math mode using the align environment,
\begin{align}
  2x^2 + 14x + 24 
  &= 2( x^2 + 7x + 12 ) \label{eqn:importantInfo}\\
  &= 2(x + 3)(x + 3)  \label{eqn:finalAnswer}
\end{align}
Expression \ref{eqn:finalAnswer} is my final answer.
\end{mhotexbox}
You probably noticed that each line of this environment was assigned a 
number. If you want to be able to reference a line from the environemnt 
you can provide it with a label and then reference it as 
expression \ref{eqn:importantInfo}.

It might be the case that you do not want to reference or number each 
expression in an align environment.  In this case, you can use the 
can append a star to the environment name to suppress the numbering.

\begin{mhotexbox}
  Given \( x = \cos x \) and \( y = \sin x \), we can show
  \begin{align*}
    x^2 + y^2 &= 1 \\
    \left( \cos x \right)^2 + \left( \sin x \right)^2 &= 1 \\
    \cos^2 x + \sin^2 &= 1.
  \end{align*}
\end{mhotexbox}

It is also possible to provide text between each line of your 
environment without having to use a new environment each time 
you want to move back to display style math mode.  This is 
the \verb!intertext! command.  If you want less space between
each line then you can use the \verb!shortintertext! command.

\begin{mhotexbox}
  Given \( x = \cos x \) and \( y = \sin x \), we can show
  \begin{align*}
    \shortintertext{using the Pythagorean identity}
    x^2 + y^2 &= 1 \\
    \intertext{substituting \(x \) and \(y \)}
    \left( \cos x \right)^2 + \left( \sin x \right)^2 &= 1 \\
    \shortintertext{alternative function squared form}
    \cos^2 x + \sin^2 &= 1.
  \end{align*}
\end{mhotexbox}
