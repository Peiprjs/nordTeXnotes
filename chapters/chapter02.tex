%----------------------------------------------------------------------------------------------------------------------------------------------------------%
\chapter{Physical experimentation}
%----------------------------------------------------------------------------------------------------------------------------------------------------------%
\section{Description}
\paragraph{}This first part of the experimental process took place in the school's biology laboratory. While it may not be the most ideal type of laboratory for this type of research, it is certainly adequate enough to perform a research project like this one. The process involves extracting a sample from underneath a subject's nalis by swabbing, cultivating that sample and then observing the results of said culture to determine the presence or not of Staphylococcus Aureus as part of the subject's resident bacterial flora. Each iteration of the process took less than two minutes to complete. However, all the safety measures and actions taken would need more time to be taken care of properly. 
%----------------------------------------------------------------------------------------------------------------------------------------------------------%
\section{Protocol followed}
\paragraph{}Protocol
%----------------------------------------------------------------------------------------------------------------------------------------------------------%
\section{Bill of materials}
\paragraph{}The materials used, as well as the quantities used can be found in the following table. On the left, laboratory equipment and, on the right, regents and consumables used.
%----------------------------------------------------------------------------------------------------------------------------------------------------------%
\section{Risk assessment and prevention}
\paragraph{}Staph is considered a Biosecurity Level (BSL) 2 pathogenic bacteria. This means that the it is associated with a human disease that can pose a moderate human health hazard. In a laboratory where BSL-2 pathogens are handled, regular lab rules should be followed (mechanical pipetting only, hand washing, prohibiting the consumption of food and drinks in the lab, proper PPE use...), as well as avoiding splashes or aerosols, biohazard warning signs present on all material used, as well as proper surface and material disinfection via the use of autoclave or alternative decontamination method.[source: book] The risks associated with this bacteria were assessed following the protocol designated by the World Health Organisation [cite], and proper security measures were followed at all times when handling biohazardous material. No incidents occurred during the research part of this project, and the protocol defined previous to the start was followed to a T.