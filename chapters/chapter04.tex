%----------------------------------------------------------------------------------------------------------------------------------------------------------%
\chapter{Conclusions}
%----------------------------------------------------------------------------------------------------------------------------------------------------------%
\epigraph{Our reliance on the validity of a scientific conclusion depends ultimately on a judgment of coherence; and as there can exist no strict criterion for coherence, our judgment of it must always remain a qualitative, non-formal, tacit, personal judgment.}{\textit{Michael Polanyi}}
%----------------------------------------------------------------------------------------------------------------------------------------------------------%
\section{Experimental conclusions}
This study has concluded that the prevalence of \emph{Staphylococcus aureus} in our high school is 48,8\%, 150\% of the expected results. As explored previously, this could mostly be due to climate change or antibiotic abuse, however there may also be other reasons for why this is happening.\newline It has also managed to produce a 3D printed figure of the bacteriophage that can help eradicate \emph{Staphylococcus aureus}, regardless of antibiotic resistance, as well as discovering how the proteins bind to the surface of a bacterium, injects its DNA and then proceeds to its hosted reproduction.
%----------------------------------------------------------------------------------------------------------------------------------------------------------%
\section{Bibliographic conclusions}
While \emph{Staphylococcus aureus} is a dangerous bacteria given the right conditions, most of the times, the immunitary system can get rid of it before it becomes too large of a problem. However, in some cases, when the entire body gets infected and the infection stops being localized then that's when there is a problem. There are several strains of \emph{Staphylococcus aureus}, classified by their resistance to antibiotics: MSSA (sensitive to meticillin), MRSA (resistant to meticillin), VISA (intermediate resistance to vancomycin) and VRSA (resistant to vancomycin). While there is no antibiotic that can deal with VRSA, an alternative in the form of a bacteriophage virus, P-68, of the order of the \emph{Caudovirales}
%----------------------------------------------------------------------------------------------------------------------------------------------------------%
\section{Strengths and weaknesses}
This research was not without its strengths, but neither was it without its weaknesses.
\paragraph{Strengths} The protocol was adapted fairly well to the environment it was run in, and no incidents took place during the realization of the experimentation.
\paragraph{Weaknesses} While the Agar plates used were definitely adequate for the purpose they were used for, a much more appropriate growth medium called Baird-Parker (BP) could've been used.
%----------------------------------------------------------------------------------------------------------------------------------------------------------%
\section{Possible improvements}
This research could've been improved by running an antibiogram on the samples, thus checking for antibiotic resistance. While this is fairly safe if adequate protections are taken, it is yet another point that could fail and result in a biosafety incident.\newline
It could have also been improved by obtaining even a larger sample of the population, in order to get an even more significative result