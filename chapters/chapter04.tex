%=-=-=-=-=-=-=-=-=-=-=-=-=-=-=-=-=-=-=-=-=-=-=-=-=-=-=-=-=-=-=-=-=-=- CHAPTER 03
\chapter{Theorem Environments}

%=-=-=-=-=-=-=-=-=-=-=-=-=-=-=-=-=-=-=-=-=-=-=-=-=-=-=-=-=-=-=-=-=-=-=-= SECTION
\section{Definition}

%=-=-=-=-=-=-=-=-=-=-=-=-=-=-=-=-=-=-=-=-=-=-=-=-=-=-=-=-=-=-=-==-=-= DEFINITION
\begin{definition}[Definition Of Subtraction]
\label{0000}\index{Definition!0000}
Given \( a \) and \( b \) are real numbers, then\cite{wolframMathworld-list}
\cite{Olson2021}
\begin{align*}
    a + (-b) &= a - b.
\end{align*}
\end{definition}
%=-=-=-=-=-=-=-=-=-=-=-=-=-=-=-=-=-=-=-=-=-=-=-=-=-=-=-=-=-=-=-=- END DEFINITION

%=-=-=-=-=-=-=-=-=-=-=-=-=-=-=-=-=-=-=-=-=-=-=-=-=-=-=-=-=-=-=-=-=-=-=-= SECTION
\section{Examples}

%=-=-=-=-=-=-=-=-=-=-=-=-=-=-=-=-=-=-=-=-=-=-=-=-=-=-=-=-=-=-=-=-=-=-=-= EXAMPLE
\begin{example}
\label{0001}\index{Example!0001}
Solve the equation \( x + 4 = 7 \) for all \( x \in \setZ \).
\end{example}
% solution
\begin{solution}
The solution set is \( x = \set{3} \).
\end{solution}
%=-=-=-=-=-=-=-=-=-=-=-=-=-=-=-=-=-=-=-=-=-=-=-=-=-=-=-=-=-=-=-=-=-= END EXAMPLE

%=-=-=-=-=-=-=-=-=-=-=-=-=-=-=-=-=-=-=-=-=-=-=-=-=-=-=-=-=-=-=-=-=-=-=-= SECTION
\section{Theorems}

%=-=-=-=-=-=-=-=-=-=-=-=-=-=-=-=-=-=-=-=-=-=-=-=-=-=-=-=-=-=-=-==-=-=-=-=- AXIOM
\begin{axiom}[Axiom Of Simple Mathematical Induction]
\label{0005}\index{axiom!0005}
Let \(P(n)\) be a proposition, where \( n \in \setZp \) .  If we can
\begin{enumerate}
    \item show the basis step is TRUE by showing \(P(1)\) is TRUE and 
    \item show the inductive step is TRUE by showing for each 
    \( r \in \setZp \), whenever \( P(r) \) is TRUE , then \( P(r + 1) \) is
    TRUE, where \( P(r) \) is called the \defw{induction hypothesis}, 
\end{enumerate}
then by the \defw{axiom of mathematical induction} the proposition is proved.
        
The axiom of simple mathematical induction cannot be proven as it is part of the definition of \(\setZp\).
        
\end{axiom}
%=-=-=-=-=-=-=-=-=-=-=-=-=-=-=-=-=-=-=-=-=-=-=-=-=-=-=-=-=-=-=-==-=-=- END AXIOM

The axiom of simple mathematical induction\ref{0005} cannot be proven as it is 
included as part of the definition of the set of positive integers, 
\( \setZp \).

%=-=-=-=-=-=-=-=-=-=-=-=-=-=-=-=-=-=-=-=-=-=-=-=-=-=-=-=-=-=-=-==-=- PROPOSITION
\begin{proposition}[Product of Common Base Powers]
\label{0003}\index{Proposition!0003}
Given \( b^m \) and \( b^n \), then
\begin{align*}
    b^m \cdot b^n &= b^{m + n}.
\end{align*}
\end{proposition}
\begin{proof}
    Coming Soon!
\end{proof}
%=-=-=-=-=-=-=-=-=-=-=-=-=-=-=-=-=-=-=-=-=-=-=-=-=-=-=-=-=-=-=-= END PROPOSITION

%=-=-=-=-=-=-=-=-=-=-=-=-=-=-=-=-=-=-=-=-=-=-=-=-=-=-=-=-=-=-=-==-=- PROPOSITION
\begin{theorem}[Pythagorean Theorem]
    \label{0004}\index{Theorem!0004}
    Given a right-angled triangle \( ABC \) with sides \( a, b \) and \( c \),
    where \( c \) is the hypotenuse, then
    \begin{align*}
        a^2 + b^2 &= c^2.
    \end{align*}
    \end{theorem}
    \begin{proof}
        Someday soon!
    \end{proof}
%=-=-=-=-=-=-=-=-=-=-=-=-=-=-=-=-=-=-=-=-=-=-=-=-=-=-=-=-=-=-=- END PROPOSITION

