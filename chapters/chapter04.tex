%----------------------------------------------------------------------------------------------------------------------------------------------------------%
\chapter{Conclusions}
%----------------------------------------------------------------------------------------------------------------------------------------------------------%
\epigraph{Our reliance on the validity of a scientific conclusion depends ultimately on a judgement of coherence; and as there can exist no strict criterion for coherence, our judgement of it must always remain a qualitative, non-formal, tacit, personal judgement.}{\textit{Michael Polanyi}}
%----------------------------------------------------------------------------------------------------------------------------------------------------------%
\section{Bibliographic conclusions}
While \emph{Staphylococcus aureus} is a dangerous bacterium given the right conditions However, most times, the immune system can get rid of it before it becomes too large of a problem. However, in some cases, when the entire body gets infected and the infection stops being localized, then that's when there is a problem. There are several strains of \emph{Staphylococcus aureus}, classified by their resistance to antibiotics: MSSA (sensitive to methicillin), MRSA (resistant to methicillin), VISA (intermediate resistance to vancomycin) and VRSA (resistant to vancomycin). While there is no antibiotic that can deal with VRSA, an alternative in the form of a bacteriophage virus, P-68, of the order of the \emph{Caudovirales}
%----------------------------------------------------------------------------------------------------------------------------------------------------------%
\section{Experimental conclusions}
This study has concluded that the prevalence of \emph{Staphylococcus aureus} in our high school is 48,8\%, one and a half times the expected results. As explored previously, this could mostly be due to climate change or antibiotic abuse, however there may also be other reasons for why this is happening. Our initial hypothesis, which was that the prevalence of epitelial \emph{Staphylococcus aureus} in school would be at around 30\% was found out to be false. Instead, the experimentation found a prevalence of the bacteria being 46,5\%. This value, 1,5 times larger than the one expected, may probably have not come from experimental error, as the procedure was followed rigorously, and the risk of contamination was mitigated to levels with which we could confidently say that no plates were subjects of cross-contamination between batches.\newline
%----------------------------------------------------------------------------------------------------------------------------------------------------------%
\section{Strengths and weaknesses}
This research was not without its strengths, but neither was it without its weaknesses.
\paragraph{Strengths} The protocol was adapted fairly well to the environment it was run in, and no incidents took place during the realization of the experimentation. The cost of the experimentation was relatively cheap, taking into account that reagents in microbiology can quickly get expensive. Reliability was also high, and the questions were answered, hypotheses verified and refused.
\paragraph{Weaknesses} While the Agar plates used were definitely adequate for the purpose they were used for, a much more appropriate growth medium called Baird-Parker (BP) could've been used. A much more adequate and comfortable lab environment would also have been a very welcomed improvement.
%----------------------------------------------------------------------------------------------------------------------------------------------------------%
\section{Possible improvements}
This research could've been improved by running an antibiogram on the samples, thus checking for antibiotic resistance. While this is fairly safe if adequate protections are taken, it is yet another point that could fail and result in a biosafety incident.\newline
It could have also been improved by obtaining even a larger sample of the population, in order to get an even more significant result. The bacteria could have been sequenced, allowing us to trace back the bacterium one by one, comparing it to locations where a similar strain had been found, tracing back its evolution and possible path followed around the world.\newline
Another interesting factor to be looked at could be the familial relationship between subjects that tested positive. It may be possible that there is some genetic character that causes members of a same family to have a predisposition of having \emph{Staphylococcus aureus} under their nails, as compared to subjects not from the same family.
