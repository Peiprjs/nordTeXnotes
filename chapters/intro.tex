%----------------------------------------------------------------------------------------------------------------------------------------------------------%
\chapter{Prologue}
%----------------------------------------------------------------------------------------------------------------------------------------------------------%
\epigraph{Good writing starts strong. Not with a cliché, not with a banality, but with a contentful observation that provokes curiosity.}{\textit{Stephen King}}
%----------------------------------------------------------------------------------------------------------------------------------------------------------%
\paragraph{}Our extremities are extremely important, and have been so since the dawn of time. They allow us not only to move about, but to socialize and tend to others, to express ourselves using art, to take care of ourselves, and many more things that we think are a given and thus, pay no attention to. That's why, when parts  of them start to fail, humanity looked at its best friend, the brain, and asked it to develop a solution for this problem. And so it did: humans had created the artificial implant. A lot of people who weren't able to walk now could. However, a few months later, doctors realized that the zone around where they implanted the piece was warm to the touch and was starting to swell and redden: typical symptoms of an infection. So, they treated it with broad-spectrum antibiotics, in hopes that this simple treatment fixed the ailment while scientists applied Koch's Pustulates in order to find the cause.. And it did… that time. A few years later, after thousands of such infections, one patient stopped responding to antibiotics. And then another. And then a hundred more. The first antibiotic-resistant bacteria were born. So, after much studying, doctors, pharmaceutics, and biologists joined forces and developed a new compound to help fight it. And it helped until, again, there were some cases in which it didn't. So the doctors, with the help of biotechnologists, turned to a fairly old line of research that had been abandoned as soon as penicillin was discovered: the little bacteria-eating vira appropriately named \emph{bacteriophages}.\newline
A few years later, summer of 2021, I was accepted into a program at the Barcelona Autonomous University, which aimed to divulge microbiology and biotechnology to a group of 50 biology-loving students. That's where I learned in more detail about bacteria, and how a microbiology/biotechnology lab functioned. I fell in love with the discipline at first sight. I wondered how this magical-sounding research works, and so, I thought it was a great fit for my Extended Essay.

\paragraph{}This Extended Essay has the objective of studying bibliographically the effects of \emph{Staphylococcus aureus} on the human body, as well as the ways humanity has developed to defeat it. Experimentally, it has two main objectives, and several secondary ones: mainly, I want to find out the natural prevalence of Staphylococcus Aureus among my fellow schoolmates; as well as finding out the shape of the bacteriophage used to fight the most resistant strains of it, how it binds to bacteria and then reproduces using it; thus prompting the following questions \emph“What is the prevalence of \emph{Staphylococcus aureus} in our school” and \emph“What structures does the corresponding bacteriophage use to detect and bind to cells”, to which my hypotheses are ”About 30\%” and ”Some type of sensor protein”. Secondarily, this has the objectives of improving my lab etiquette and fluidity; allowing me to improve my protocols, how I follow them in the lab and how I deal with problems that may arise; forcing me to learn how to work with limited resources; practising my tinction and microscope use; allowing me to practice analogue translation of DNA into amino acids; and last but not least, teaching me and helping me perfect how to use AlphaFold.

\paragraph{}This study requires taking samples from live human subjects. This is a one-off sampling process: the subjects are required only once. The results are then communicated to the subjects via e-mail or by being delivered a physical piece of paper. They are informed previously on the process they are going to go through, as well as the purpose of the experiment. Each subject must read and agree to two documents: an informed consent which explains everything about the experiment\footnote{See annex 1} and a GDPR notice which documents the use of their data as well as an expected timeline for data anonymization and destruction\footnote{See annex 2}. All participants were screened to be over the age of 16, in order to ease the process and require no previous authorization by parental figures on the data collection. The experimentation followed has no effect on the subjects\cite{WhatGDPREU2018}.
\paragraph{}Since bacteria were used, some aspects of the experiment must be clarified and discussed. Previously to starting the experiment, I read profusely the WHO's Laboratory Biosafety Manual and Associated Monographs (4th Edition)\cite{worldhealthorganizationLaboratoryBiosafetyManual2020} in order to find ways to mitigate any possible risk. During the experimental phases, there were no accidents or incidents. All plates were accounted for and controlled closely. No person other than me was allowed to come in contact with a plate that had been cultivated or with any used but not disinfected auxilliary material. The cultivated plates were considered Biosecurity Level 2. All possibly infected material was disposed of taking into account the risks that the bacteria in question posed, using fresh bleach.
\paragraph{}Before starting the experimentation, and following the guidelines dictated by the IBO about the EE, I had a talk with my coordinator in order to solidify the fact that there was no alternative to taking cutaneous samples from human beings, as well as a discussion on bacteria and the risks that this experiment implies.
