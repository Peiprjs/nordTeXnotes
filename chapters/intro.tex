%----------------------------------------------------------------------------------------------------------------------------------------------------------%
\chapter{Introduction}
%----------------------------------------------------------------------------------------------------------------------------------------------------------%
\epigraph{Good writing starts strong. Not with a cliché, not with a banality, but with a contentful observation that provokes curiosity.}{\textit{Stephen King}}
%----------------------------------------------------------------------------------------------------------------------------------------------------------%
A few years ago, in summer of 2021, I was accepted into a program at the Barcelona Autonomous University, aimed to divulge microbiology and biotechnology to a group of 50 biology-loving students. That's where I learned bacteria in detail, as well as how a microbiology/biotechnology lab functions. I fell in love with the discipline at first sight. I wondered how research in this field works. I took one of the experiments we did, and decide to expand it for my EE.\newline %----------------------------------------------------------------------------------------------------------------------------------------------------------%
This extended essay has as primary objective answering these two questions:
\begin{center}"\emph{What is the prevalence of \emph{Staphylococcus aureus} in our school}?"\end{center}
\begin{center}"\emph{Is the prevalence of \emph{Staphylococcus aureus} affected by gender or age?}"\end{center}
And its secondary ones include studying bibliographically \emph{Staphylococcus aureus}, improving my lab etiquette and protocol-making, and allowing me to practice microbiology techniques.
%----------------------------------------------------------------------------------------------------------------------------------------------------------%
\paragraph{}This study required taking samples from  human subjects. The results were, when available, communicated to the subjects via e-mail. They were informed previously on the process they would go through, as well as the purpose of the experiment. Each subject had to read read and agree to two documents: an informed consent which explains everything about the experiment\footnote{See annex 1}, and a GDPR notice which documents the use of their data as well as an expected timeline for its destruction\cite{WhatGDPREU2018}\footnote{See annex 2}. The experimentation followed lead to no effect on the subjects.
\paragraph{}Since bacteria were used, some aspects of the experiment had to be clarified and discussed. Previously to starting to design the protocol, I read the WHO's Laboratory Biosafety Manual and Associated Monographs (4Th Edition)\cite{worldhealthorganizationLaboratoryBiosafetyManual2020}, to mitigate or eliminate any possible risk. During the experimental phases, there were no incidents. All plates were accounted for. No person other than me was allowed to come in contact with any of the Petri dishes, nor with any used but not yet disinfected auxiliary material. The cultivated plates were considered Biosecurity Level 2. All possibly infected material was disinfected following the WHO recommendations. Following the IBO EE guidelines, I talked with my coordinator in order to solidify the fact that there was no alternative to sampling from humans, as well as to consider the risks that this experiment implied because of bacteria.
