%----------------------------------------------------------------------------------------------------------------------------------------------------------%
\chapter{Prologue}
%----------------------------------------------------------------------------------------------------------------------------------------------------------%
\epigraph{Good writing starts strong. Not with a cliché, not with a banality, but with a contentful observation that provokes curiosity.}{\textit{Stephen King}}
%----------------------------------------------------------------------------------------------------------------------------------------------------------%
A few years ago, in summer of 2021, I was accepted into a program at the Barcelona Autonomous University, aimed to divulge microbiology and biotechnology to a group of 50 biology-loving students. That's where I learned bacteria in detail, as well as how a microbiology/biotechnology lab functions. I fell in love with the discipline at first sight. I wondered how research in this field works, and so, I thought it was a good topic to develop for my Extended Essay.

\paragraph{}This Extended Essay has the objective of studying bibliographically the effects of \emph{Staphylococcus aureus} on the human body, as well as the ways humanity has developed to defeat it. Experimentally, it has one main objective, and several secondary ones: mainly, I want to find out the natural prevalence of Staphylococcus Aureus among my fellow schoolmates. Secondarily, I want to improve my lab etiquette and fluidity; to improve my protocol-making, how I follow them in the lab and how I deal with problems that may arise from them; to learn how to work with limited resources; and to practice my staining and microscope use. The research question I will follow is \emph``What is the prevalence of \emph{Staphylococcus aureus} in our school`` to which my hypothesis is ``About 30\%``. 

\paragraph{}This study requires taking samples from  human subjects. This is a one-off sampling process: the subjects are required only once. The results are then communicated to the subjects via e-mail or by being delivered a physical piece of paper. They are informed previously on the process they are going to go through, as well as the purpose of the experiment. Each subject must read and agree to two documents: an informed consent which explains everything about the experiment\footnote{See annex 1} and a GDPR notice which documents the use of their data as well as an expected timeline for data anonymisation and destruction\footnote{See annex 2}. All participants were screened to be over the age of 16, in order to ease the process and require no previous authorisation by parental figures on the data collection. The experimentation followed has no effect on the subjects\cite{WhatGDPREU2018}.
\paragraph{}Since bacteria were used, some aspects of the experiment must be clarified and discussed. Previously to starting the experiment, I read profusely the WHO's Laboratory Biosafety Manual and Associated Monographs (4Th Edition)\cite{worldhealthorganizationLaboratoryBiosafetyManual2020} in order to find ways to mitigate any possible risk. During the experimental phases, there were no accidents or incidents. All plates were accounted for and controlled closely. No person other than me was allowed to come in contact with a plate that had been cultivated or with any used but not disinfected auxiliary material. The cultivated plates were considered Biosecurity Level 2. All possibly infected material was disposed of taking into account the risks that the bacteria in question posed, using fresh bleach.
\paragraph{}Before starting the experimentation, and following the guidelines dictated by the IBO about the EE, I had a talk with my coordinator in order to solidify the fact that there was no alternative to taking cutaneous samples from human beings, as well as a discussion on bacteria and the risks that this experiment implies.
