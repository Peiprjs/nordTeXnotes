%----------------------------------------------------------------------------------------------------------------------------------------------------------%
\chapter{Theoretical context}
%----------------------------------------------------------------------------------------------------------------------------------------------------------%
\epigraph{Good writing starts strong. Not with a cliche, not with a banality, but with a contentful observation that provokes curiosity.}{\textit{Alexander Chee}}
=======
\epigraph{Good writing starts strong. Not with a cliché, not with a banality, but with a contentful observation that provokes curiosity.}{\textit{Stephen King}}
>>>>>>> Stashed changes
%----------------------------------------------------------------------------------------------------------------------------------------------------------%
\paragraph{}Our extremities are extremely important, and have been so since the dawn of time. They allow us not only to move about, but to socialise and tend to others, to express ourselves using art, to take care of ourselves, and many more things that we think are a given and thus, pay noa attention to. That's why, when parts  of them start to fail, humanity looked at its best friend, the brain, and asked it to develop a solution for this problem. And so it did: humans had created the artificial implant. A lot of injured people now could walk. However, a few months later, doctors realised that the zone around where they implanted the piece was hot to the touch and was starting to swell: typical symptoms of an infection. So, they treated it with broad-spectrum antibiotics, in hopes that this simple treatment fixed the ailment. And it did... that time. A few years later, after thousands of such infections, one patient stopped responding to antibiotics. And then another. And then a hundred more. The first antibiotic-resistant bacteria were born. So, after much studying, doctors, pharmaceutics and biologists developed a new compound to help with it. And it helped, until there were some cases in which it didn't. So the doctors, with the help of biotechnologists, turned to a fairly old line of research that had been dropped as soon as penicillin was discovered: the little bacteria-eating vira.\newline
A few years later, summer of 2021, I had been accepted into a program at the Barcelona Autonomous University, which aimed to divulge microbiology and biotechnology to a group of 50 biology-loving students. That's where I learned in more detail about bacteria, and how a microbiology/biotechnology lab functioned; and I fell in love with the discipline. 

\section{Introduction}
This research project will study the effects of \emph{Staphylococcus aureus} on the human body, as well as the ways humanity has developed to defeat it. Experimentally, the goal is to find the prevalence amongst the students in the school and recreate virtually a virus used to defeat this bacteria. The bacteria that are going to be sampled will also be tinted and observed under an optical microscope.\newline
The investigation question is ``\emph{What is the prevalence of} Staphylococcus aureus\emph{in the high school?}``
