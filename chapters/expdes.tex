%----------------------------------------------------------------------------------------------------------------------------------------------------------%
\chapter{Experimental design}
%----------------------------------------------------------------------------------------------------------------------------------------------------------%
\epigraph{It is common sense to take a method and try it; if it fails, admit it frankly and try another. But above all, try something.}{\textit{Anthony Burgess}}
%----------------------------------------------------------------------------------------------------------------------------------------------------------%
\paragraph{}This Extended Essay has the objective of studying bibliographically the effects of \emph{Staphylococcus aureus} on the human body, as well as the ways humanity has developed to defeat it. Experimentally, it has one main objective, and several secondary ones: mainly, I want to find out the natural prevalence of Staphylococcus Aureus among my fellow schoolmates. Secondarily, I want to improve my lab etiquette and fluidity; to improve my protocol-making, how I follow them in the lab and how I deal with problems that may arise from them; to learn how to work with limited resources; and to practice my staining and microscope use. The research question I will follow is "\emph{What is the prevalence of \emph{Staphylococcus aureus} in our school}?" to which my hypothesis is \emph{``About 30\%``}. I would also like to know the answer to: "\emph{Is the prevalence of \emph{Staphylococcus aureus} affected by gender or age}" to which my hypothesis is \emph{``No``}.  The variable I will study is the presence or not of the bacterium in question on different subjects, and compare it against their characteristics (such as approximate age and gender). I'll keep constant the culture medium, as well as the culture temperature and humidity.
