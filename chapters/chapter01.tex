%=-=-=-=-=-=-=-=-=-=-=-=-=-=-=-=-=-=-=-=-=-=-=-=-=-=-=-=-=-=-=-=-=-= CHAPTER 01
\chapter{Introduction to mhotext}

% =-=-=-=-=-=-=-=-=-=-=-=-=-=-=-=-=-=-=-=-=-=-=-=-=-=-=-=-=-=-=-=-=-=-= SECTION
\section{Purpose}

This template is designed for International Baccalaureate\texttrademark\, 
students who are looking to write an extended essay or an internal assessment 
using \LaTeX.  The template is based on the KOMA Script scrbook class and styled
using the custom style file 
\texttt{mhotext.sty}\footnote{\url{https://github.com/mholson/mhoDotFiles/texmf/}}, 
which was inspired by Evan Chan's code
used create Napkin\footnote{\url{http://web.evanchen.cc/napkin.html}}.

While I have done my best to ensure this template meets the submission
requirement of the IB, I cannot guarantee your document will meet the 
these standards.  It is the responsibility of the student to ensure that any 
work submitted is formatted correctly by ensuring they are following the 
official IB documentation and the advice giving by their supervising teacher.

While there is extensive online support and documentation on how to write
a document using \LaTeX\,, it can still be overwhelming for anyone who is just
getting started.  I have been using \TeX\, since 2005 and I am still
learning!  Part of this documentation will be used to provide you with
enough knowledge to get you up and writing as quickly as possible.


% =-=-=-=-=-=-=-=-=-=-=-=-=-=-=-=-=-=-=-=-=-=-=-=-=-=-=-=-=-=-=-=-=-=-= SECTION
\section{Where to Begin?}

Since \TeX\, is a programming language we are going to need a text editor
to write our code and a compiler to transform that code into a pdf document.
To keep things simple, we are going to be using an online service called 
Overleaf\footnote{\url{https://overleaf.com}}, which will provide us with a 
modern text editor and a range of compilers that we can use all within the 
comfort of our favorite web browser.

Now, rather than think of our document as one file that we edit, such as a
MS Word or Google document, we are going to think of it as an object called a
project which will be made up files and folders. For us, there is going to be
one main file called \texttt{main.tex} that will make use of other files
within the project to build a pdf of your extended essay or internal assessment.

% =-=-=-=-=-=-=-=-=-=-=-=-=-=-=-=-=-=-=-=-=-=-=-=-=-=-=-=-=-=-=-=-=-=-= EXAMPLE
\begin{example}
  Rather than drag an image into MS Word document, we will now
  upload the image to our project and insert it into our pdf document using
  a line of code.
\end{example}

It doesn't stop there!  All the template files necessary to format your document
will also exist in your project.  Don't worry, these files will be tucked away
in a folder called \texttt{texmf}.

We are starting to get a little far ahead of ourselves here.  All that really
matters at this point is that we have an object called a project that is made up
of files and folders.

% =-=-=-=-=-=-=-=-=-=-=-=-=-=-=-=-=-=-=-=-=-=-=-=-=-=-=-=-=-=-=-=-=-=-= SECTION
\section{Project Structure}
Our project is going to be made up of three folders and one main file called
\texttt{main.tex}. Let's describe the contents of those three folders.
\begin{description}
  \item[texmf] You probably will not do very much with the files located
        in this folder as these are the template files that are responsible for
        formatting the content of your essay. In most cases, you can change
        the formatting elements of your document right from the
        \texttt{main.tex} file.  It is also good practice to not edit these
        files directly, so that your theme can easily be upgraded if necessary.
  \item[assets] An asset is a file that you are going to include in your
        document.  In most cases, the only files that you will be including
        in your document will be images and thus you can upload your images
        directly to this folder.  You can have as many subfolders within your
        assets folder, but again this folder should be good enough to upload
        your images into.
  \item[chapters] This folder is really optional.  You could technically write
        your document all within the \texttt{main.tex} file; however, it might
        be beneficial to break your document up into chapters and
        write these in separate files.  If you are not going to be structuring
        your paper using chapters, then you should probably just write your
        document in the \texttt{main.tex} file.
\end{description}

Like I said before, I want to get you up and writing as quickly as possible.
We are going to get right into editing the \texttt{main.tex} file and producing
a pdf file with some content.  Action!

% =-=-=-=-=-=-=-=-=-=-=-=-=-=-=-=-=-=-=-=-=-=-=-=-=-=-=-=-=-=-=-=-=-=-= SECTION
\section{The Main File}
The \texttt{main.tex} file is where all the action takes place.  It is in this
file where you will write your content or at least link to the content found in
other files in your project.  This file can be as simple or as complicated as 
you would like it to be.  I have provided a skeleton template of this file in
your template project.  So let's start looking at the code.

For us the file is going to start of with multiple lines of text that are going
to be ignored by the compiler, which means it will not be included in our pdf
output.

% =-=-=-=-=-=-=-=-=-=-=-=-=-=-=-=-=-=-=-=-=-=-=-=-=-=-=-=-=-=-=-=-=-=-= SECTION
\section{texmf - The Template Files}

Most of you will never need to look in this folder; however, I thought I would
provide a short summary of what files are included and how they are being used
to format your document.

% =-=-=-=-=-=-=-=-=-=-=-=-=-=-=-=-=-=-=-=-=-=-=-=-=-=-=-=-=-=-=-=-=-=-= SECTION
\section{Requirements}

All the file dependencies outside the scope of a typical \TeX\, distribution
are available from my \texttt{texmf} repository hosted on
Github\footnote{\url{https://github.com/mholson/mhoTexmf/tree/main/texmf}}.
The Overleaf\texttrademark version will come with batteries included in the 
\texttt{texmf} directory, which in include.  
\begin{description}
    \item[\texttt{maabook.cls}] the main template file.
    \item[\texttt{mhotext.cls}] customizations made the to main template file.
    \item[\texttt{mhotheorems.sty}] styling custom theorem environments
    \item[\texttt{mhocolorthemenord}] custom color theme based on Nord Theme.
    \item[\texttt{mhocodestyle.sty}] styling code via the listings package.
    \item[\texttt{mhomath.sty}] some of my custom math commands.  
\end{description}


% =-=-=-=-=-=-=-=-=-=-=-=-=-=-=-=-=-=-=-=-=-=-=-=-=-=-=-=-=-=-=-=-=-=-= SECTION
\section{Getting Started}
% =-=-=-=-=-=-=-=-=-=-=-=-=-=-=-=-=-=-=-=-=-=-=-=-=-=-=-=-=-=-=-=-=- DEFINITION
\begin{definition}[Code Comment]
  Any code that is written in your file that is ignored by the compiler is
  called \defw{commented text}.  All characters on the same line following
  a \verb!%! symbol will be regarded as commented text and ignored.
\end{definition}

Comments are great for including additional information, making temporary 
changes to your document (without having to erase) and documenting your
code for others to follow --- including yourself!

\begin{dispExample}
% This line would be ignored by the compiler.
This line is read by the compiler.

The compiler is reading % and now it is ignoring.
\end{dispExample}
One of the benefits of writing in \LaTeX\, is the ability to write comments
that are not typeset.  Any text following a \verb!%! character will be 
ignored.  Not only is this ideal for leaving comments and ideas, but it is 
als a way to make temporary changes to your document.



\begin{mhotexbox}
% This code will be ignored
This text will be visible % while this text will be ignored.
\end{mhotexbox}
Much of your document will be filled with commands that take the form 
. In fact the first non-comment line of your document will be the command 
that defines the document template.
\begin{dispListing}
\documentclass[a4paper]{maatext}

\end{dispListing}
more text here
