%=-=-=-=-=-=-=-=-=-=-=-=-=-=-=-=-=-=-=-=-=-=-=-=-=-=-=-=-=-=-=-=-=-=- CHAPTER 01
\chapter{Introduction}

% =-=-=-=-=-=-=-=-=-=-=-=-=-=-=-=-=-=-=-=-=-=-=-=-=-=-=-=-=-=-=-=-=-=-= SECTION
\section{Purpose}

This template is designed for International Baccalaureate\texttrademark\, 
students who are looking to write an extended essay or an internal assessment 
using \LaTeX.  The template is based on the KOMA Script scrbook class and styled
using the custom style file 
\texttt{mhotext.sty}\footnote{\url{https://github.com/mholson/mhoDotFiles/texmf/}}, 
which was inspired by Evan Chan's code
used create Napkin\footnote{\url{http://web.evanchen.cc/napkin.html}}.

While I have done my best to ensure this template meets the submission
requirement of the IB, I cannot guarantee your document will meet the 
these standards.  It is the responsibility of the student to ensure that any 
work submitted is formatted correctly by ensuring they are following the 
official IB documentation and the advice giving by their supervising teacher.

While there is extensive online support and documentation on how to write
a document using \LaTeX\,, it can still be overwhelming for anyone who is just
getting started.  I have been using \TeX\, since 2005 and I am still
learning!  Part of this documentation will be used to provide you with
enough knowledge to get you up and writing as quickly as possible.


% =-=-=-=-=-=-=-=-=-=-=-=-=-=-=-=-=-=-=-=-=-=-=-=-=-=-=-=-=-=-=-=-=-=-= SECTION
\section{Where to Begin?}

Since \TeX\, is a programming language we are going to need a text editor
to write our code and a compiler to transform that code into a pdf document.
To keep things simple, we are going to be using an online service called 
Overleaf\footnote{\url{https://overleaf.com}}, which will provide us with a 
modern text editor and a range of compilers that we can use all within the 
comfort of our favorite web browser.

Now, rather than think of our document as one file that we edit, such as a
MS Word or Google document, we are going to think of it as an object called a
project which will be made up files and folders. For us, there is going to be
one main file called \texttt{main.tex} that will make use of other files
within the project to build a pdf of your extended essay or internal assessment.

% =-=-=-=-=-=-=-=-=-=-=-=-=-=-=-=-=-=-=-=-=-=-=-=-=-=-=-=-=-=-=-=-=-=-= EXAMPLE
\begin{example}
  Rather than drag an image into MS Word document, we will now
  upload the image to our project and insert it into our pdf document using
  a line of code.
\end{example}

It doesn't stop there!  All the template files necessary to format your document
will also exist in your project.  Don't worry, these files will be tucked away
in a folder called \texttt{texmf}.

We are starting to get a little far ahead of ourselves here.  All that really
matters at this point is that we have an object called a project that is made up
of files and folders.

While there is extensive online support and documentation for \LaTeX\,, it 
can still be overwhelming to a beginner.  I do hope the following information 
and examples will give students a sufficient knowledge base from which they 
can beging writing and extend their skills. 

%=-=-=-=-=-=-=-=-=-=-=-=-=-=-=-=-=-=-=-=-=-=-=-=-=-=-=-=-=-=-=-=-=-=-=-= SECTION
\section{Requirements}

%=-=-=-=-=-=-=-=-=-=-=-=-=-=-=-=-=-=-=-=-=-=-=-=-=-=-=-=-=-=-=-=-=-=-=-= SECTION
\section{Getting Started}

One of the benfits of writing in \LaTeX\, is the ability to write comments
that are not typeset.  Any text following a \lstinline{%} character will be 
ignored.  Not only is this ideal for leaving comments and ideas, but it is 
als a way to make temporary changes to your document.
\lstset{style=mhotexcode}
\begin{lstlisting}[belowskip=-2 \baselineskip]
% This code will be ignored
This text will be visible % while this text will be ignored.
\end{lstlisting}
Much of your document will be filled with commands that take the form 
. In fact the first non-comment l
ine of your document will be the command that defines the document template.
\begin{lstlisting}
\documentclass[a4paper]{maatext}
\end{lstlisting}
