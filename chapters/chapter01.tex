%=-=-=-=-=-=-=-=-=-=-=-=-=-=-=-=-=-=-=-=-=-=-=-=-=-=-=-=-=-=-=-=-=-= CHAPTER 01
\chapter{Introduction to IB\TeX}

% =-=-=-=-=-=-=-=-=-=-=-=-=-=-=-=-=-=-=-=-=-=-=-=-=-=-=-=-=-=-=-=-=-=-= SECTION
\section{Purpose}

This template is designed for International Baccalaureate\texttrademark\ 
students who are looking to write an extended essay or an internal assessment 
that is formatted using \LaTeX.  The template is based on the \KOMAScript\ scrbook
class and further customized using a number of custom style files, including  
\texttt{mhotext.sty}\footnote{\url{https://github.com/mholson/mhoDotFiles/texmf/}}
that is inspired by the code Evan Chan used create 
Napkin\footnote{\url{http://web.evanchen.cc/napkin.html}}. 

While I have done my best to ensure this template meets the submission
requirement of the IB, I cannot guarantee your document will meet the 
these standards.  It is the responsibility of the student to ensure that any 
work submitted to the IB for grading is formatted correctly by ensuring they 
are following the official IB documentation and the advice giving by their 
supervising teacher.

While there is extensive online support and documentation on how to publish 
using the \LaTeX\ programming language, it can still be overwhelming for anyone
who is getting started.  I have been navigating the \TeX\ ecosystem since 2005
and still learning.  I want to share some of the knowledge and ideas that I have
accumulated along the way, so that you can get writing
as quickly as possible.  You should be focused on writing an amazing 
manuscript and then put this template take care of the typesetting for you.

Before going any further, I feel that it is important that we establish that this 
documentation is written for high school students with little to no prior \TeX\ 
knowledge. While you should be able to typeset your document without an 
understanding of that \TeX\ is, how it works and a concise overview
of \TeX\ is and how it has developed over the past 40\textsuperscript{+} years,
I would recommend that you jump to chapter 06 before going any further
If you are not quite sure what the difference is between \LaTeX\ and \TeX , 
then I would highly recommend that you jump to chapter 06 for concise overview
of \TeX\ is and how it has developed over the past 40\textsuperscript{+} years.
While it is not imperative that you have an understanding of how \TeX\ has 
evolved, it does bring some useful context to how it works today.  
 

% =-=-=-=-=-=-=-=-=-=-=-=-=-=-=-=-=-=-=-=-=-=-=-=-=-=-=-=-=-=-=-=-=-=-= SECTION
\section{What is \LaTeX?}


so let's try to distill things down the necessities with a 
little historical context to guide us.  The 
programming language \TeX\ was created by Donal Knuth in the late
1970s to automate the process of formatting text ready for publication, 
a process called typesetting.  It worked by the user creating a \texttt{foo.tex} 
text file formatted with commands collectively called plain\TeX. This file would 
then be interpreted by a computer program called the \TeX\ engine, which would 
output a device independent file, \texttt{foo.dvi}, that could be sent to printer or 
converted to another file format such as \texttt{foo.ps} or \texttt{foo.pdf}.

However, in the 1980s Knuth strategically decided to that no new features would
be added to the \TeX engine as Knuth valued the benefits of an engine that
could produce the same output over new features being added. This has held true
up to today with the exception of one last feature being added in 1989 resulting
in \TeX3 \cite{GuideManyFlavours}.  With each bug fix update to \TeX3 the version number 
converges to \( \pi \) and in 2021 it reached version 3.14159265\textbf{3}\cite{CTANPackageLshort}.

Since \TeX\ is a programming language we are going to need a text editor
to write our code and a compiler to transform that code into a pdf document.
To keep things simple, I would highly recommend using  
Overleaf\footnote{\url{https://overleaf.com}}, which will provide you with  
both a modern text editor and a range of compilers that can all be used from
the comfort of your favorite web browser.  If you want to compile your code
on your own machine, then you will need to learn how to do that elsewhere.

Before you start coding, I would highly recommend that you spend some time
reading The Not So Short Introduction to \LaTeXe\ \cite{CTANPackageLshort}. 
Maybe start with at least the first two chapters just to give you some history
and context to \TeX.  Keep a link to this documentation close by as it 
makes for a great reference manual.

This document has been written so that you should be able to get into the 
weeds if you feel the need, but more importantly ensure that you can get 
writing almost immediately.  So now for a very generalized overview of \TeX.

You are probably familiar with the idea of writing your document all in one 
file using a word processor like MS Word or Google Documents. A shift in mindset
will be needed as your document will be now be an object called a project made 
up of multiple files and folders. 

\section{The Big Idea}

Basically, we are going to input code stored in a \texttt{main.tex} file into
a compiler that will generate an output file.  The code read by the compiler is 
written in using rhe written using the \TeX\ 
programming language, 

\section{Project Structure}
\begin{figure}[ht]

\dirtree{%
.1 \cnordOne{\faWarehouse} ~Project.
.2 \cnordThree{\faFileCode} ~main.tex.
.2 \cnordFifteen{\faFileCode} ~references.tex.
.2 \cnordEleven{\faFilePdf} ~main.pdf.
.2 \cnordTen{\faFolder} ~assets.
.3 \cnordTen{\faFolder} ~img.
.4 \cnordSeven{\faFileImage} ~exampleImage001.jpg.
.4 \cnordSeven{\faFileImage} ~exampleImage002.jpg.
.4 \cnordEleven{\faFilePdf} ~exampleImage003.pdf.
.2 \cnordTen{\faFolder} ~texmf.
.3 \cnordFourteen{\faFileCode} ~mhotext.sty.
.3 \cnordFourteen{\faFileCode} ~mhocolorthemenord.sty.
.3 \cnordFourteen{\faFileCode} ~mhomath.sty.
.3 \cnordFourteen{\faFileCode} ~mhocodestyle.sty.
.3 \cnordFourteen{\faFileCode} ~mhotheorems.sty.
.2 \cnordTen{\faFolder} ~chapters.
.3 \cnordThree{\faFileCode} ~chapter01.tex.
.3 \cnordThree{\faFileCode} ~chapter02.tex.
.3 \cnordThree{\faFileCode} ~chapter02.tex.
}
\caption{Example of a project file structure.}
\end{figure}


% =-=-=-=-=-=-=-=-=-=-=-=-=-=-=-=-=-=-=-=-=-=-=-=-=-=-=-=-=-=-=-=-=-=-= EXAMPLE
\begin{example}{Including an image.}
  When inserting an image into a MS Word document, you would insert the image 
  and it would become part of the document and you would no longer see that
  image file associated with your document.  
  
  When inserting an image into a \LaTeX ~document, we upload a copy of the 
  image into our project and then insert a line of code that will place that
  image into our document.
\end{example}

While the number of files can grow when writing a \LaTeX ~document, there are 
some pretty awesome advantages such as only having to modify or update an 
image file and the changes will be visible in your document following the 
next compilation of your code.

I am starting to get ahead of myself here. Let's 

It doesn't stop there!  All the template files necessary to format your document
will also exist in your project.  Don't worry, these files will be tucked away
in a folder called \texttt{texmf}.



% =-=-=-=-=-=-=-=-=-=-=-=-=-=-=-=-=-=-=-=-=-=-=-=-=-=-=-=-=-=-=-=-=-=-= SECTION

Our project is going to be made up of three folders and one main file called
\texttt{main.tex}. Let's describe the contents of those three folders.
\begin{description}
  \item[texmf] You probably will not do very much with the files located
        in this folder as these are the template files that are responsible for
        formatting the content of your essay. In most cases, you can change
        the formatting elements of your document right from the
        \texttt{main.tex} file.  It is also good practice to not edit these
        files directly, so that your theme can easily be upgraded if necessary.
  \item[assets] An asset is a file that you are going to include in your
        document.  In most cases, the only files that you will be including
        in your document will be images and thus you can upload your images
        directly to this folder.  You can have as many subfolders within your
        assets folder, but again this folder should be good enough to upload
        your images into.
  \item[chapters] This folder is really optional.  You could technically write
        your document all within the \texttt{main.tex} file; however, it might
        be beneficial to break your document up into chapters and
        write these in separate files.  If you are not going to be structuring
        your paper using chapters, then you should probably just write your
        document in the \texttt{main.tex} file.
\end{description}

Like I said before, I want to get you up and writing as quickly as possible.
We are going to get right into editing the \texttt{main.tex} file and producing
a pdf file with some content.  Action!

% =-=-=-=-=-=-=-=-=-=-=-=-=-=-=-=-=-=-=-=-=-=-=-=-=-=-=-=-=-=-=-=-=-=-= SECTION
\section{The Main File}
The \texttt{main.tex} file is where all the action takes place.  It is in this
file where you will write your content or at least link to the content found in
other files in your project.  This file can be as simple or as complicated as 
you would like it to be.  I have provided a skeleton template of this file in
your template project.  So let's start looking at the code.

For us the file is going to start of with multiple lines of text that are going
to be ignored by the compiler, which means it will not be included in our pdf
output.

% =-=-=-=-=-=-=-=-=-=-=-=-=-=-=-=-=-=-=-=-=-=-=-=-=-=-=-=-=-=-=-=-=-=-= SECTION
\section{texmf - The Template Files}

Most of you will never need to look in this folder; however, I thought I would
provide a short summary of what files are included and how they are being used
to format your document.

% =-=-=-=-=-=-=-=-=-=-=-=-=-=-=-=-=-=-=-=-=-=-=-=-=-=-=-=-=-=-=-=-=-=-= SECTION
\section{Requirements}

All the file dependencies outside the scope of a typical \TeX\, distribution
are available from my \texttt{texmf} repository hosted on
Github\footnote{\url{https://github.com/mholson/mhoTexmf/tree/main/texmf}}.
The Overleaf\texttrademark version will come with batteries included in the 
\texttt{texmf} directory, which in include.  
\begin{description}
    \item[\texttt{maabook.cls}] the main template file.
    \item[\texttt{mhotext.cls}] customizations made the to main template file.
    \item[\texttt{mhotheorems.sty}] styling custom theorem environments
    \item[\texttt{mhocolorthemenord}] custom color theme based on Nord Theme.
    \item[\texttt{mhocodestyle.sty}] styling code via the listings package.
    \item[\texttt{mhomath.sty}] some of my custom math commands.  
\end{description}


% =-=-=-=-=-=-=-=-=-=-=-=-=-=-=-=-=-=-=-=-=-=-=-=-=-=-=-=-=-=-=-=-=-=-= SECTION
\section{Getting Started}
% =-=-=-=-=-=-=-=-=-=-=-=-=-=-=-=-=-=-=-=-=-=-=-=-=-=-=-=-=-=-=-=-=- DEFINITION
\begin{definition}[Code Comment]
  Any code that is written in your file that is ignored by the compiler is
  called \defw{commented text}.  All characters on the same line following
  a \verb!%! symbol will be regarded as commented text and ignored.
\end{definition}

Comments are great for including additional information, making temporary 
changes to your document (without having to erase) and documenting your
code for others to follow --- including yourself!

\begin{dispExample}
% This line would be ignored by the compiler.
This line is read by the compiler.

The compiler is reading % and now it is ignoring.
\end{dispExample}
One of the benefits of writing in \LaTeX\, is the ability to write comments
that are not typeset.  Any text following a \verb!%! character will be 
ignored.  Not only is this ideal for leaving comments and ideas, but it is 
als a way to make temporary changes to your document.



\begin{mhotexbox}
% This code will be ignored
This text will be visible % while this text will be ignored.
\end{mhotexbox}
Much of your document will be filled with commands that take the form 
. In fact the first non-comment line of your document will be the command 
that defines the document template.
\begin{dispListing}
\documentclass[a4paper]{maatext}

\end{dispListing}
more text here
