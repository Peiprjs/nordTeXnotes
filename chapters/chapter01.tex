%----------------------------------------------------------------------------------------------------------------------------------------------------------%
\chapter{Introduction}
%----------------------------------------------------------------------------------------------------------------------------------------------------------%
\section{An overview of the study}

%----------------------------------------------------------------------------------------------------------------------------------------------------------%
\section{Bacteria and bacterial infections}
\paragraph{Bacteria} are prokaryote organisms, generally single-celled, which are part of the Monera animal kingdom. Their sizes range from between \SI{30}{\micro\metre} and \SI{100}{\micro\metre} and are ubiquitous\footnote{Ubiquitous: found everywhere} organisms. This form of life is believed to be the first one to have ever appeared on Earth, as well as the one responsible for the oxigen-rich atmosphere the Earth currently has. Some species are hard to culture in a laboratory environment, but generally, those that can be cultured in a controlled environment are grown in agar plates\cite{MicrobioMed}. \newline
Agar is used as a place to grow bacteria due to the fact that it is indigestible for the majority of bacteria, yet it keeps them humid and, together with growth mediums, such as Lysogeny Broth, bacteria thrive in this environment, allowing them to proliferate and create colonies, which can be seen to the naked eye. Sometimes, together with the growth medium, additives such as mannitol salt are added. These are used to improve or impede bacterial growth, modify their conditions so they develop differently or as an identification tool. For example, \emph{Staphyloccus Aureus} ferments it, producing acid, which in turn decolorates the plate from red to yellow.
\paragraph{Pathogenic bacteria} are bacteria that have the ability to cause disease\footnote{''A disease is a particular abnormal condition that negatively affects the structure or function of all or part of an organism, and that is not immediately due to any external injury.''\cite{dorlands:001}}. These are not the most common type of bacteria, as the majority of them are either harmless or benefitial to the human body through symbiosis, such as the bacteria that help with digestion in the stomach\footnote{citation needed}.

%----------------------------------------------------------------------------------------------------------------------------------------------------------%
\section{The enemy: Staphylococcus aureus}
\paragraph{}\emph{Staphylococcus Aureus} (also known as Staph) is a GRAM-positive bacteria, the most virulent and studied of its genus\footnote{citation needed, got to check the proper terminology}. Some of its distinctive characteristics include having a very thick glycopeptide wall, which allows it to withstand extreme temperatures and osmotic pressures, therefore rendering most classic methods of food conservation (such as cooking, smoking, freezing or salting\footnote{citation needed}) completely useless against said bacteria;  a protein A capsid, which binds to many eukaryote organism. It's an extremely resistant (and thus ubiquitous) bacteria. It can be found in human skin and mucotic surfaces (such as the mouth or the nose), as well as in certain foods such as ham (cooked or curated), eggs, raw and cooked dough, as well as in poultry. This is due to the fact that it has a large and thick glycopeptide capsule, which protects it from extreme temperatures as well as osmotic pressure. Thus, it can survive highly salty, extremely cold and extremely hot environments.
\paragraph{}
%----------------------------------------------------------------------------------------------------------------------------------------------------------%
\section{The enemy's weapons}
\paragraph{}I'll write this back at the UDG library with the book  I used to make that thing. If I can't, I will just find the scanned pages and work my way backwards from there.
%----------------------------------------------------------------------------------------------------------------------------------------------------------%
\section{Our weapons}
\paragraph{}I'll write this back at the UDG library with the book  I used to make that thing. If I can't, I will just find the scanned pages and work my way backwards from there.
%----------------------------------------------------------------------------------------------------------------------------------------------------------%

