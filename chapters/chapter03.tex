%----------------------------------------------------------------------------------------------------------------------------------------------------------%
\chapter{Computational experimentation}
%----------------------------------------------------------------------------------------------------------------------------------------------------------%
\section{Description}
\paragraph{}This second experiment looks at computing the shape of a \emph{Staphylococcus Aureus} P-68 bacteriophage starting from its DNA sequence. This will analyse the process of protein translation, as well as looking at the process of using AlphaFold\cite{jumperHighlyAccurateProtein2021}. To compute the secondary, tertiary and quaternary structures of the proteic parts of the virus we will use the Google CoLaboratory version of AlphaFold\cite{GoogleColaboratoryAlpha1970}, as it allows for far more computer power than available on a simple laptop.
%----------------------------------------------------------------------------------------------------------------------------------------------------------%
\section{Protocol followed}
\begin{enumerate}[label=\arabic*)]
\item Convert the DNA sequence of one of the proteins to an mRNA sequence.
\item Convert the mRNA sequence to an aminoacid sequence of one of the proteins.
\item Repeat for all the proteins that conform the virus.
\item Assemble the bacteriophage.
\end{enumerate}
%----------------------------------------------------------------------------------------------------------------------------------------------------------%
\section{Results and analysis}
[Model was erroneous]