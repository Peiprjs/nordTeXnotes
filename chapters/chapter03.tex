%----------------------------------------------------------------------------------------------------------------------------------------------------------%
\chapter{Computational experimentation}
%----------------------------------------------------------------------------------------------------------------------------------------------------------%
\epigraph{Remember that all models are wrong; the practical question is how wrong do they have to be to not be useful.}{\textit{Norman Richard Draper}}
%----------------------------------------------------------------------------------------------------------------------------------------------------------%
\section{Description}
\paragraph{}This second experiment looks at computing the shape of a \emph{Staphylococcus aureus} P-68 bacteriophage starting from its DNA sequence. This will analyse the process of protein translation, as well as looking at the process of using AlphaFold\cite{jumperHighlyAccurateProtein2021}. To compute the secondary, tertiary and quaternary structures of the proteic parts of the virus we will use the Google CoLaboratory version of AlphaFold\cite{GoogleColaboratoryAlpha1970}, as it allows for far more computer power than available on a simple laptop.
%----------------------------------------------------------------------------------------------------------------------------------------------------------%
\section{Protocol followed}
The protocol followed is very simple, as AlphaFold, the Artificial Intelligence model used does the hardest part of the work for us:
\begin{enumerate}[label=\arabic*)]
\item Convert the DNA sequence of one of the proteins to an mRNA sequence, by taking into account the fact that there's base complimentariety.
\item Convert the mRNA sequence to an aminoacid sequence of the proteins by using the genetic universal code table.
\item Feed the aminoacid sequences to AlphaFold, obtain the models for each protein.
\item Assemble the bacteriophage. 
\item Print, polish and paint a 3D physical model of the bacteriophage to illustrate better how it functions.
\end{enumerate}
The AlphaFold Jupiter notebook (can be found in 
%----------------------------------------------------------------------------------------------------------------------------------------------------------%
\section{Results and analysis}

