\documentclass[fontsize=12pt,twoside=semi,openright,numbers=noenddot,parskip=half]{scrbook}
\usepackage{scrhack}\usepackage{mhotext}\usepackage{epigraph, geometry, graphicx, adjustbox, wrapfig, mathtools, booktabs, siunitx, setspace, subcaption, booktabs,multicol,dirtree,notoccite,xcolor,pifont,float}\usepackage[T1]{fontenc}\usepackage[UKenglish]{babel}\usepackage{gentium}\usepackage{biblatex}[sorting=none,backend=biber,style=apa,citestyle=numeric-comp]\usepackage{pdfpages}\usepackage{enumitem}\graphicspath{{./assets/}}\onehalfspacing\usepackage{nopageno}
%----------------------------------------------------------------------------------------------------------------------------------
\begin{document}
\chapter*{Abstract}
\paragraph{}This paper explores how a type of bacterium, \emph{Staphylococcus aureus}, affects an infected human host, as well as its physical structure and how it relates to its virulence and survivability. It also explores the drugs scientists have developed to fight it, as well as how it generated resistance to said treatments. Finally, it also explores how scientists are working to develop tools to fight resistant infections, and not only of this kind. It is a field that is highly important in the present day, as multiresistant bacterial infections are only going to rise in the succeeding years, due to the antibiotic agent abuse that has been going on for decades.
\paragraph{}Experimentally, this paper found out the prevalence of said bacterium in our school by taking samples from students and cultivating them in MSA plates, which will indicate whether a specific subject has a natural prevalence of \emph{Staphylococcus aureus}. It also uses the sequenced material genetic code found in \emph{P-68}, the bacteriophage specific for \emph{Staphylococcus aureus}, to find its primary, secondary, tertiary, and quaternary proteic structures by using artificial intelligence and human thinking.
\paragraph{}The research found a worrying increase in natural prevalence, getting results that were 150\% over the expected; resulting in a prevalence percentage at around 50\%. I also tried to find the reason for these increased results, discussing the fact that it may be linked to antibiotic abuse or climate change. However, more data is needed to draw a proper correlation. The results obtained from the DNA sequence were used to print a functional 3D figure,  used to help understand how the bacteriophage functions.
\end{document}
