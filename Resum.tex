\documentclass[fontsize=12pt,twoside=semi,openright,numbers=noenddot,parskip=half]{scrbook}
\usepackage{scrhack}\usepackage{mhotext}\usepackage{epigraph, geometry, graphicx, adjustbox, wrapfig, mathtools, booktabs, siunitx, setspace, subcaption, booktabs,multicol,dirtree,notoccite,xcolor,pifont,float}\usepackage[T1]{fontenc}\usepackage[UKenglish]{babel}\usepackage{gentium}\usepackage{biblatex}[sorting=none,backend=biber,style=apa,citestyle=numeric-comp]\usepackage{pdfpages}\usepackage{enumitem}\graphicspath{{./assets/}}\onehalfspacing\usepackage{nopageno}
%----------------------------------------------------------------------------------------------------------------------------------
\begin{document}
\chapter*{Resum}
\paragraph{}Aquest treball de recerca explora com un tipus de bacteri, \emph{Staphylococcus aureus}, afecta un hoste humà infectat, així com la seva estructura física i la manera en què aquesta es relaciona amb la seva virulència i supervivència. També explora els fàrmacs que els científics han desenvolupat per combatre'l, així com la manera com va generar resistència a aquests tractaments. Finalment, també explora com els científics estan treballant per crear eines per combatre les infeccions multiresistents, i no només d'aquest tipus de bacteri. Es tracta d'un camp de gran importància en l'actualitat, ja que les infeccions bacterianes multiresistents només faran que augmentar en els pròxims anys, per culpa de l'abús d'antibiòtics que s'està fent des de fa dècades, tant per part de metges com de pacients.
\paragraph{}Experimentalment, aquest treball va identificar la prevalença d'aquest bacteri al nostre institut, prenent mostres dels alumnes i cultivant-les en plaques MSA, que van indicar si un subjecte concret té una prevalença natural de \emph{Staphylococcus aureus}. Utilitzant el codi genètic material seqüenciat que es va trobar de la seqüenciació d'una mostra de  \emph{P-68}, el bacteriòfag específic de \emph{Staphylococcus aureus}, es van calcular les seves estructures proteiques primàries, secundàries, terciàries i quaternàries mitjançant la intel·ligència artificial i el pensament humà.
\paragraph{}La investigació va trobar un augment preocupant de la prevalença natural, obtenint resultats que van ser un 150\% per sobre de l'esperat; resultant en un percentatge de prevalença al voltant del 50\%. Vaig intentar trobar la raó d'aquests resultats sobrealçats, discutint el fet que podria estar relacionat amb l'abús d'antibiòtics o el canvi climàtic. Tanmateix, calen més dades per establir una correlació adequada. Els resultats aconseguits per part de la seqüència de DNA es van utilitzar per imprimir una figura funcional en 3D, utilitzada per ajudar a entendre com funciona el bacteriòfag.
\end{document}
